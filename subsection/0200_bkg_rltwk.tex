%%%%%%%%%%%% WIP. Needs extensie rephrases!!!!!

\subsection{Linux package management}

% Add a short summary for this subsection
A modern software platform has extremely large number of packages with complex inter-package dependencies, which include a lot of shared libraries and daemon services having complicated dependencies and divided into smaller ``sub-packages''.
As a result, a package management system is required to handle complex inter-package dependencies consistently and efficiently.


\textit{Mancinelli et al.}~\cite{11Mancinelli2006ASE} have proposed to handle dependencies between packages automatically.
Most of its prior approaches have focused on declaring forward dependencies (i.e., \texttt{Requires} or \texttt{BuildRequires} in RPM systems) and following them.
This work~\cite{11Mancinelli2006ASE} tries to automatically discover undeclared dependencies by inspecting build scripts.
Caixa Magica and Mandriva Linux distributions use this method.


%%%%%%%%%%%% WIP. Needs extensie rephrases!!!!! CANNOT UNDERSTAND!
Requirements for future software build systems including choices, benefits, and challenges of a well-designed build process are described in~\cite{12Smith2011SBSBook}.
It surveys the tools and techniques for building software and how things may go wrong.
They conjecture that inadequate build systems can dramatically deteriorate the productivity; a subpar build system may incur bad dependencies, false compile errors, failed software images, slow compilation, and excessive labors.
It explains how to increase the performance and scalability of a build system. 


%%%%%%%%%%%% WIP. Needs extensie rephrases!!!!! CANNOT UNDERSTAND!
\textit{Galindo et al.}~\cite{13Galindo2010DebianPR} suggest realistic variability models to evaluate dependencies, a major and traditional problem of Linux communities.
They suggest that the package dependency language of Debian is a variability language and provide a mapping from the language to propositional formulas analyzing dependencies with SAT solvers.
They address other variability dependency languages of open source communities as well, which increases the availability of realistic variability models up to 20,000 packages for the Software Product Line community.
However, they do not detect anomalies as conditionally dead packages or redundancies.


\textit{Cosmo et al.}~\cite{14DiCosmo2008PUF} point out why the upgrade problems faced by Free and Open Source Software (FOSS) distributions have characteristics not easily found elsewhere.
They provide periodic snapshots of a whole software platform, which can mitigate OS upgrade problems along with disk partitioning.
They survey current countermeasures to such upgrade failures, argue that they are not satisfactory, and sketch alternative solutions.
They focus only on applying fingerprinting techniques to cluster maintainer scripts of Debian distribution to get a clear view of all their use cases.


\textit{Adoption of Academic Tools}~\cite{15abate2017debian} presents an overview of 10 years of research in this field and the process leading to the adoption of our tools in a FOSS community.
They presented the check tools such as \textit{distcheck} and \textit{buildcheck}, which scan all the packages in a Debian distribution to identify installability issues.
They focused on the Debian distribution and in particular they looked at the issues arising during the distribution lifecycle: ensuring buildability of source packages, detecting packages that cannot be installed and bootstrapping the distribution on a new architecture. 


\textit{Schroeder}~\cite{16Schroeder2007SatSolverURL} depicts the package dependency solver library that is called SAT solver.
This project has been started in May 2007 when the ZYpp community has decided to use a database to speed up installation.
It offers an efficient file and memory representation for complex dependency relations and package repositories.
The SAT solver of Libzypper is a port from the red carpet solver, which is to update packages of a running system.
In addition to SAT solver, they provide ad-hoc mechanisms to discover some of undeclared dependencies and an audit function for weak dependencies.


\textit{Zypper}~\cite{20Zypper_URL} is a package manager that installs, updates, and removes packages and manages repositories.
It is especially useful for managing software packages remotely or with shell scripts.
With Zypper, we can easily update the distribution.
Besides, we can update the software platform in run-time.
Note that Tizen is fully compatible with Zypper although it is not included in most traditional Tizen profiles.

\subsection{Build systems of Tizen}

% Add a short summary for this subsection
We describe build and release systems of Tizen, a software platform for IoT/edge devices, general consumer electronics, mobile phones, wearable devices, or even autonomous driving vehicles.
The standard packaging for Tizen is RPM~\cite{foster2003red}, which is also standard for OpenSUSE, RedHat, and Fedora.
We do not discuss Debian packages~\cite{blackman2000debian} although it is one of the two major standards along with RPM in Linux communities.
Note that the expression power of inter-package dependencies of Debian is not higher than that of RPM; Debian package dependencies may be expressed with RPM dependencies.

\textit{Git Build System (GBS)}~\cite{17GBS2014TizenURL} is a build and packaging tool for Tizen platform development.
It generates tarballs, builds sources, and packages binaries from Git repositories.
GBS also does local unit tests, provides build and test sandboxes, submits code to the build infrastructure, OBS.
Package maintainers may use GBS to maintain their upstream branches or forks, or to prototype packages not included in Tizen mainline.


\textit{MIC Image Creator (MIC, originated from Meego Image Creator)}~\cite{18MIC2014TizenURL} builds software platform images for Tizen.
MIC creates images of different types, including live CD images, live USB images, raw images for KVM~\cite{21Kivity07kvm}, loop images for IVI platforms, and filesystem images for chrooting.
Users can enter into the generated images with MIC. MIC changes the apparent root directory with a \textit{chroot} mechanism for the current running processes.
Note that MIC uses Zypper~\cite{20Zypper_URL} to resolve package dependencies during image creation.


\textit{Open Build Service (OBS)}~\cite{19OBS_URL} is a general build, release, and distribution system for various target platforms in an automatic, consistent, distributed, and reproducible way.
OBS releases software for a wide range of operating systems and hardware architectures with extensible web interfaces and APIs.
It is an open and complete distribution development platform that provides a transparent infrastructure for the development of Linux distributions, used by openSUSE and Tizen.
OBS supports Fedora, Debian, Ubuntu, RedHat, and many distributions.
Like GBS, OBS builds binaries in a sandbox to ensure the consistency.
OBS may create binary packages in varying formats including RPM~\cite{22foster2002red}, DEB, and many others.
The created packages can be released and deployed via repositories compatible with package managers.


