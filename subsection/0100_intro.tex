Traditionally, software platforms for consumer electronics (e.g., Tizen, Android, and iOS) have clear target device profiles: phones, wearable devices, tablets, or TVs.
Such a profile defines common API sets, features, and requirements across different products of the profile.
However, the wide spread of IoT and edge devices has introduced a new challenge: the long tail of the IoT~\cite{1Want2015EnablingTI}, which, in turn, incurs following significant challenges to software platforms:
\begin{itemize}
\item Each IoT device might require its own customized software platform: \textit{configurability}.
\item Even with a shared source code repository, various software platforms are built with different configurations, incurring extremely high build workloads: \textit{build explosion}.
\end{itemize}


Both challenges have already incurred significant developmental costs with traditional smart devices.
\textit{Build explosion} enforces developers and infrastructure to build and test software packages repeatedly for each profile.
Then, \textit{build explosion} attributes to the lack of \textit{configurability} by incurring high developmental costs to platform developers, which increase further with the long tail of the IoT.
Even if we have only 10 different IoT devices, not 100, we need to develop with 10 different build configurations for each commit and the build infrastructure suffers from additional x 10 workloads.
This is especially unacceptable with the long tail-ness; each IoT device type has little revenue although the whole IoT device portfolio may have huge revenue.
We cannot afford linearly increasing development costs with the increasing number of IoT device types.
Tizen platform developers have been already overloaded with the workloads multiplied by the number of profiles and suffered from the lengthy build task queues of the overloaded build infrastructure.
Thousands of build tasks had waited for several hours in task queues during busy hours despite of a lot of powerful build servers.


\textit{Configurability} is deteriorated not only by \textit{build explosion}, but also by the difficulties in composing a customized software platform with software packages.
There are well-established package management systems (e.g., Advanced Package Tool (APT) for DEB packages~\cite{blackman2000debian} and Zypper/DNF for RPM packages~\cite{foster2003red}), which allow to choose software packages without understanding inter-package dependencies.
However, developers still need to choose packages required by their software platform from the thousands of candidates, which require deep understandings of the software platform from device drivers and system software to applications.
Because such developers are rare, prototyping a lot of varying devices and their customized software platforms cannot be accomplished with the traditional and well-established tools.
Therefore, we need new tools that allow novice developers (developers without such deep understandings) to compose properly customized software platforms.


We address \textit{build explosion} first in order to enable \textit{configurability} because we should be able to build software packages for various devices with affordable workloads and to provide a single unified binary package repository of candidate packages to be chosen from.
We address \textit{build explosion} by unifying Tizen build procedures and binary repositories, enforcing every profile to share a single repository of binary packages, not only the source codes: Tizen:Unified project.
In order to unify Tizen, we introduce new rules for platform developers and plugin architectures with new dependency rules, enforced by build systems and Linux packaging systems, which are upstreamed to open source communities as well so that the new rules comply with the standard.
Thanks to the cooperation from platform developers and the communities, we have fully unified Tizen projects and repositories since Tizen 4.0 in 2017.


Unifying binary repositories mitigates the build workload not only for new IoT devices, but also for the traditional smart devices.
Tizen project (Tizen:Unified in \url{https://build.tizen.org}) no longer builds five times for five profiles, but builds only once for all device types, which is expected to reduce the workload of infrastructure roughly to one fifth.
With Tizen:Unified, peak build task waiting queue size has dropped from thousands to tens \cite{2Ham2017TDC}.


This work applies new infrastructure designed to promote \textit{configurability} along with Building Blocks of Tizen after Tizen:Unified.
Building Block definitions~\cite{3TizenWikiBB,4LeeKeynote2017TDC} provide easy-to-configure components to build a customized software platform maintained by product managers.
Building Blocks are meta packages with hierarchical structures that provide easy-to-understand descriptions.
A Building Block, existing as an RPM package, does contain any files, but has relational information of software packages and other Building Blocks; thus, it is a meta package.
Tizen Image Creator (TIC)~\cite{2Ham2017TDC} offers a web user interface to compose a software platform with Building Blocks and individual packages.
Tizen also offers a public web service for novice IoT developers, Craftroom~\cite{5CraftroomURL}, a simplified and beautified variation of TIC.
TIC and Craftroom provide new customized software platforms for IoT developers within minutes, not hours or days.


We design and implement both projects, Tizen:Unified and Tizen:Configurability, refactor software packages, restructure build procedures and rules, which enables on-the-fly platform customization.
Build and packaging rules introduced in this work are either within the Linux standard (RedHat/OpenSUSE) or upstreamed to the community.
This work significantly reduces build and test workloads of conventional devices and enables Tizen for wider ranges of device types with significantly reduced workloads of both developers and infrastructure.
This work is applied to Tizen 4.0 (2017) and 5.x (2018/2019), enabling various new products and prototypes, ranging from home appliances and IoT prototypes to autonomous driving systems and on-device AI platforms.