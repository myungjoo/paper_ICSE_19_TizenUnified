
We have unified build projects and binary repositories of the software platform for various device types, improving the developmental efficiency, and proposed the concepts of Building Blocks and the highly configurable platform.
In the due course, we have restructured the Tizen by refactoring hundreds of packages and implementing key infrastructures to support Configurability with Building Blocks.
This work is successfully released via Tizen 4.0 and succeeded to Tizen 5.0 without any developmental overheads to keep the proposed mechanisms intact.


The productivity of both platform developers and infrastructures has improved significantly and Tizen has become capable of providing software platforms for various IoT and edge devices on-the-fly.
The overall productivity of platform developers is improved by reducing turnaround time from code writing to integration and deployment and by reducing the number of binary packages to be created and tested for each source repository.
According to the analysis in May, 2017, by Tizen team~\cite{2Ham2017TDC}, the number of non-base packages built for a full build has been drastically reduced to 968 from 3,483.
Moreover, after the full migration of build systems to Amazon Web Services (AWS), this work saves the cost of running AWS by reducing the number of build tasks.
The improved configurability has allowed creating software platforms for various IoT devices, enabling IoT projects for Tizen and prototypes including autonomous driving systems and on-device artificial intelligence embedded systems.